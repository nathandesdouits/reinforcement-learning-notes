\section{Exercises}

\subsubsection{~}

\textit{In the comparison in fig~2.1, which method will perform best in the long run in terms of cumulative reward, and cumulative probability of selecting the best action?}

Because of the law of large numbers, in the long run we have $Q_t(a) \approx q(a)$ for $\epsilon$-greedy methods, since each actions will have been sampled a very large number of times.
This is of course not true for the greedy method which correctly sample only one action, the one it always choses.
For $\epsilon$-greedy method, in the long run we select the action $A^* = \argmax_a q(a)$ $(1 - \epsilon)$ of the time and a random action $\epsilon$ of the time.
There doesn't seem to be any exact function that gives the expectation of the maximum of $n$ iid normal variables, I could only find inequalities for large $n$...
I computed the value for $n = 10$ using 1 million series of 10 iid normal variables, and got a value of $\approx 1.54$
The average reward when selecting action $A^*$ is $\mathbb{E}[q(A^*)] \approx 1.54$.
Meanwhile, let $Z$ be the reward when selection a random action, and $Y \sim \mathcal{N}(0,1)$ the noise term added to $q(a)$ when computing the reward.
We have:
\begin{align*}
\mathbb{E}[Z] &= \mathbb{E}[\mathbb{E}[q(a)] + Y]
              &= \mathbb{E}[q(a)] + \mathbb{E}[Y]
              &= 0,
\end{align*}
since both random variables $q(a)$ and $Y$ follow a standard normal distribution.
All in all, for $\epsilon$-greedy method, the average expected reward is:
\begin{align*}
\mathbb{E}[\bar{R}] = \epsilon \mathbb{E}[Z] + (1 - \epsilon) \mathbb{E}[q(A^*)]
                    \approx (1 - \epsilon) \times 1.54
\end{align*}
Which means 1.39 for $\epsilon = 0.1$ and 1.52 for $\epsilon = 0.01$.

For the true greedy method, the first action $A^\dagger$ for which the reward gets over 0 gets chosen everytime (at first approximation).
Thus we need to find $\mathbb{E}[q(a)| q(a) + Y > 0]$.
I could not manage to find a seemingly correct formula unfortunately (but this one should not be too hard for a statistician), so I once again computed an estimate for this number for $n = 10$, and came up with $\approx 0.56$.
This looks a bit weird since in fig~2.1 the average reward of the greedy method is around $1$.
