\section{Exercises}

\paragraph{Exercise 2.1}
\textit{In the comparison in fig~2.1, which method will perform best in the long run in terms of cumulative reward, and cumulative probability of selecting the best action?}

Because of the law of large numbers, in the long run we have $Q_t(a) \approx q(a)$ for $\epsilon$-greedy methods, since each actions will have been sampled a very large number of times.
This is of course not true for the greedy method which correctly sample only one action, the one it always choses.
For $\epsilon$-greedy method, in the long run we select the action $A^* = \argmax_a q(a)$ $(1 - \epsilon)$ of the time and a random action $\epsilon$ of the time.
There doesn't seem to be any exact function that gives the expectation of the maximum of $n$ iid normal variables, I could only find inequalities for large $n$...
I computed the value for $n = 10$ using 1 million series of 10 iid normal variables, and got a value of $\approx 1.54$
The average reward when selecting action $A^*$ is $\mathbb{E}[q(A^*)] \approx 1.54$.
Meanwhile, let $Z$ be the reward when selection a random action, and $Y \sim \mathcal{N}(0,1)$ the noise term added to $q(a)$ when computing the reward.
We have:
\begin{align*}
\mathbb{E}[Z] &= \mathbb{E}[\mathbb{E}[q(a)] + Y] \\
              &= \mathbb{E}[q(a)] + \mathbb{E}[Y] \\
              &= 0
\end{align*}
since both random variables $q(a)$ and $Y$ follow a standard normal distribution.
All in all, for $\epsilon$-greedy method, the average expected reward is:
\begin{align*}
\mathbb{E}[\bar{R}] &= \epsilon \mathbb{E}[Z] + (1 - \epsilon) \mathbb{E}[q(A^*)] \\
                    &\approx (1 - \epsilon) \times 1.54 \\
\end{align*}
Which means 1.39 for $\epsilon = 0.1$ and 1.52 for $\epsilon = 0.01$.

For the true greedy method, the first action $A^\dagger$ for which the reward gets over 0 gets chosen everytime (at first approximation).
Thus we need to find $\mathbb{E}[q(A^\dagger)] = \mathbb{E}[q(a)| q(a) + Y > 0]$.
I could not manage to find a seemingly correct formula unfortunately (but this one should not be too hard for a statistician), so I once again computed an estimate for this number for $n = 10$, and came up with $\mathbb{E}[q(A^\dagger)] \approx 0.56$.
This looks a bit weird since in fig~2.1 the average reward of the greedy method is around $1$.

For $\epsilon$-greedy method, the probability of selecting the best action when we decided to chose a random action is obviously $\frac{1}{n}$.
As for when we decide to chose the action with the highest estimated value, I get stuck.
We need to find, given $(Y, Z) \sim \mathcal{N}(0,1)^2$, the probability $Pr\{q(a) + Y > q(A^*) + Z\}$.
That honestly seems daunting to me and I would not even know where to begin anyway...
Graphically, the curve seems to plateau at around 80\% for $\epsilon = 0.1$, and should be plateauing over this value for $\epsilon = 0.01$.
Same for the greedy method, which seems to plateau at around 35\%.

\paragraph{Exercise 2.2}
\textit{Give pseudocode for a complete algorithm for the $n$-armed bandit problem. Use greedy action selection and incremental computation of action values with $\alpha = \frac{1}{k}$ step-size parameter. Assume a function $\mathit{bandit}(a)$ that takes an action and returns a reward. Use arrays and variables; do not subscript anything by the time index $t$. Indicate how the action values are initialized and updates after each reward. Indicate how the step-size parameters are set for each action as a function of how many times it has been tried.}

\begin{enumerate}
\item Initialization \\
$Q(a) \leftarrow 0$ for all $a \in \mathcal{A}$ \\
$k(a) \leftarrow 1$ for all $a \in \mathcal{A}$

\item Action selection and value update \\
Repeat \\
\-\hspace{2em} $A \leftarrow \argmax_a Q(a)$ (resolve ties randomly) \\
\-\hspace{2em} $R \leftarrow \mathit{bandit}(A)$ \\
\-\hspace{2em} $Q(A) \leftarrow Q(A) + \frac{1}{k} (R - Q(A))$ \\
\-\hspace{2em} $k(A) \leftarrow k(A) + 1$ \\
until end of epoch
\end{enumerate}

\paragraph{Exercise 2.3}
\textit{If the step-size parameters, $\alpha_k$, are not constant, then the estimate $Q_k$ is a weighted average of previously received rewards with a weighting different from that given by (2.6). What is the weighting on each prior reward for the general case, analogous to (2.6), in terms of $\alpha_k$?}

\begin{align*}
Q_{k+1} &= Q_k + \alpha_k (R_k - Q_k) \\
        &= \alpha_k R_k + (1 - \alpha_k) Q_k \\
        &= \alpha_k R_k + (1 - \alpha_k) [\alpha_{k-1} R_{k-1} + (1 - \alpha_{k-1}) Q_{k-1}] \\
        &= \alpha_k R_k + \alpha_{k-1} (1 - \alpha_k) R_{k-1} + (1 - \alpha_{k}) (1 - \alpha_{k-1}) Q_{k-1} \\
        &= Q_1 \prod_{i=1}^k (1 - \alpha_i)  + \sum_{i=1}^k \left( \alpha_i R_i \prod_{j=i+1}^{k} (1 - \alpha_j)  \right) \\
\end{align*}

The weights are then $\prod_{i=1}^k (1 - \alpha_i)$ for $Q_1$ and $\alpha_i R_i \prod_{j=i+1}^{k} (1 - \alpha_j)$ for $R_i$.

\paragraph{Exercise 2.4 (programming experiment)}
\textit{Design and conduct an experiment to demonstrate the difficulties that sample-average methods have for nonstationary problems. Use a modified version of the 10-armed testbed in which all the q(a) start out equal and then take independent random walks. Prepare plots like Figure 2.1 for an action-value method using sample averages, incrementally computed by $\alpha = \frac{1}{k}$, and another action-value method using a constant step-size parameter, $\alpha = 0.1$. Use $\epsilon = 0.1$ and, if necessary, runs longer than 1000 plays.}

See file \textit{exercise\_2\_4.py} for the Python code.
The code is composed with 3 classes (2 for bandit problem definition, Bandit and DriftBandit, and 1 for a general action-value method, EpsGreedyLearner) and 2 functions (run\_drift and experiments) to conduct the experiment.

\begin{figure}[!htb]
\centering
\includegraphics[width=0.47\columnwidth]{chap1_bandits/plot_rewards.png}
\includegraphics[width=0.47\columnwidth]{chap1_bandits/plot_bestaction.png}
\caption{\label{chap1_fig_ex24}
Left: average rewards (blue: $\alpha = \frac{1}{k}$, green: $\alpha = 0.1$, red: green/blue).
Right: probability of selecting the best action (blue: $\alpha = \frac{1}{k}$, green: $\alpha = 0.1$).
}
\end{figure}

Figure~\ref{chap1_fig_ex24}.left shows that a fixed step-size consistently outperforms the "average" step-size.
Figure~\ref{chap1_fig_ex24}.right clearly shows the difficulty of the action-value method to track a non-stationnary problem.
Indeed, the Figure 2.1 in the original manuscript reports probabilities of selecting the best action to be around 80\% in the long-run for $\epsilon = 0.1$.
For the same $\epsilon$ value, this method only reports probabilities of 50-60\% -- although this number could still increase with a higher number of time steps.


\paragraph{Exercise 2.5}
\textit{The results shown in Figure 2.2 should be quite reliable because they are averages over 2000 individual, randomly chosen 10-armed bandit tasks. Why, then, are there oscillations and spikes in the early part of the curve for the optimistic method? What might make this method perform particularly better or worse, on average, on particular early plays?}

Because of the optimistic initial value, the first $n$ actions are effectively chosen randomly among the actions that have not yet been chosen, so the probability of chosing the optimal action is around $\frac{1}{n}$.
However, for $t = n+1$, all actions have already been sampled once, and only once.
Thus, at this steps, we have $Q(a) = (1 - \alpha) Q_1 + \alpha (q(a) + X)$ for all actions, with $X \sim \mathcal{N}(0,1)$.
We can thus substract the $(1-\alpha)Q_1$ term to all action estimates without changing their ordering.
The probability of selecting the best action at $t = n+1$ is thus equal to the probability of $q(A^*) + X_{A^*} > q(a) + X_a$ for all $a \neq A^*$, which is higher than $\frac{1}{n}$.
The same is true for $2n+1$, but to a lesser extent since it is possible that the noise and/or the relative values of $q(a)$ cause the optimal action to be over- or under-sampled.
Rapidly, the noise smooths out this behavior.

\paragraph{Exercise 2.6}
\textit{Suppose you face a binary bandit task whose true action values
change randomly from play to play. Specifically, suppose that for any play the
true values of actions 1 and 2 are respectively 0.1 and 0.2 with probability 0.5
(case A), and 0.9 and 0.8 with probability 0.5 (case B). If you are not able to
tell which case you face at any play, what is the best expectation of success
you can achieve and how should you behave to achieve it? Now suppose that
on each play you are told if you are facing case A or case B (although you still
don’t know the true action values). This is an associative search task. What
is the best expectation of success you can achieve in this task, and how should
you behave to achieve it?}

In the case where it is not possible to know which case we face, we have $\mathbb{E}[R|A=1] = \mathbb{E}[R|A=1,case=A]p(case=A) + \mathbb{E}[R|A=1,case=B]p(case=B) = 0.1 \times 0.5 + 0.9 \times 0.5 = 0.5$ and $\mathbb{E}[R|A=2] = 0.5$ too.
So expectation of success is the same no matter what action is chosen and equals 0.5.

Conversely, in the case where it is possible to know the case we face, we can store the estimate of each action for each case separately.
We can use any of the tools described in this chapter to learn these estimated values.
In the end, the best expectation of success should be $0.2 \times 0.5 + 0.9 \times 0.5 = 0.55$.
